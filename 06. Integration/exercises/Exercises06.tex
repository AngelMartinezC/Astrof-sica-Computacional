\documentclass[11pt]{article}
\pagestyle{plain}
\usepackage{latexsym,exscale,amsfonts,amsmath,amssymb,array}
\usepackage{color}
\usepackage[colorlinks]{hyperref}
\setlength{\topmargin}{-2.3cm}
\setlength{\textheight}{23.8cm}
\setlength{\oddsidemargin}{-0.5cm}
\setlength{\textwidth}{17cm}
\setlength{\parindent}{0cm}
\setlength{\parskip}{.4cm}
\newcommand{\totaldiffx}{\frac{d}{dx}}
\newcommand{\pardiffx}{\frac{\partial}{\partial x}}
\newcommand{\luft}{\:\!}

\usepackage{graphicx}
\usepackage[latin1]{inputenc}
\usepackage{mathpazo}
\usepackage[T1]{fontenc}
\usepackage[comma,numbers,sort&compress]{natbib}


\begin{document}
\begin{center}
\large \bf Computational Astrophysics \rm \\
2019\\
{\small Exercises 06}
\end{center}

\begin{enumerate}
\item {\bf Newton-Cotes Integration} 
\begin{enumerate}
\item[(a)] Integrate $f(x) = \sin{x}$ from $x = 0$ to $x = \pi$
  using (1) the midpoint rule, (2) the trapezoidal rule, and (2)
  Simpson's rule. Show how the three converge with decreasing step
  size $h$ and compare their errors. (You may need to take
  special care of boundary points!)
\item[(b)] Compute the integral
\begin{equation*}
I = \int_{0}^\pi x \, \sin{x}\, dx
\end{equation*}
using (1) the midpoint rule, (2) the trapezoidal rule, and (3)
Simpson's rule. Compare the errors and show how the three converge
with decreasing step size $h$.
\end{enumerate}


\item {\bf Number Density of Electrons in an Astrophysical Environment}

In high temperature astrophysical environments ($T \gtrsim 10^9\,\mathrm{K} \sim
80\,\mathrm{keV}$), the reaction 
\begin{equation}
\gamma \leftrightarrow e^- + e^+
\end{equation}
comes into equilibrium. We will assume the
limit of $k_B T = 20\,\mathrm{MeV} \gg m_ec^2$, when the electrons
become relativistic, i.e. $E_{e^-} \sim E_{e^+} = pc$ where $p$ is the
momentum of the electrons and positrons and $c$ is the speed of light.

In such an environment, the number density of electrons/positrons is given by
\begin{equation}
n_{e^\pm} = {2 \over (2\pi\hbar)^3}\int { d^3\vec{p} \over
  e^{\beta c p} + 1} = {8\pi \over (2\pi\hbar)^3}\int_0^\infty {p^2 dp \over
  e^{\beta c p} + 1},
\end{equation}
where $\beta = 1/(k_B T)$. Making the substitution of $x= \beta p c$ to make the integral
dimensionless, we obtain

\begin{equation}
n_{e^\pm}  =  {8\pi (k_BT)^3 \over (2\pi\hbar
  c)^3}\int_0^\infty  {x^2 dx \over
  e^{x} + 1}.
\end{equation}
 
 Note that in this deduction we have considered that the chemical potential of the
electrons (and  positrons) is 0.

\begin{enumerate}
\item[(a)] Use any of the presented integration methods to determine what
  is the total number density of electrons in this environment. Try different conditions to ensure convergence. 

\item[(b)] This formula not only has the total number of
  electrons (and positrons) but also encodes the spectral distribution
  (${dn_{e^\pm} \over dE}$, i.e. $n_{e^\pm} = \int {dn_{e^\pm}\over
    dE} dE$).\\
  Such distributions are used in computational
  astrophysics all the time, but must be discretized into a finite
  number of energy groups. Hence, create energy groups with $\Delta E =
  5\,$MeV and evaluate $[dn_{e^\pm}/dE]_i = [n_{e^\pm}]_i / \Delta E$
  for each bin $i$, using any method you like. Verify your
  method by confirming that,

\begin{equation}
\sum_{i=0}^{\infty}\left[{dn_{e^\pm}\over
    dE}\right]_i \times \Delta E = n_{e^\pm}\,.
\end{equation}

Note that you will not have to calculate an infinite number of $\left[{dn_{e^\pm}\over
    dE}\right]_i$, but rather high enough until
the values are negliable, $E \sim 150\,\mathrm{MeV}$. \newline


\end{enumerate}
\end{enumerate}




\end{document}
