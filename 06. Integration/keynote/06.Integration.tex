\documentclass[11pt]{beamer}
\usetheme{Rochester}
\usepackage[utf8]{inputenc}
\usepackage{amsmath}
\usepackage{amsfonts}
\usepackage{amssymb}
\usepackage{graphicx}
%\author{}
%\title{}
%\setbeamercovered{transparent} 
%\setbeamertemplate{navigation symbols}{} 
%\logo{} 
%\institute{} 
%\date{} 
%\subject{} 
\begin{document}

\begin{frame}
\title{Computational Astrophysics}
\author{E. Larrañaga}
\institute{Observatorio Astronómico Nacional\\
Universidad Nacional de Colombia}
\titlepage
\end{frame}

\begin{frame}{Outline}
\tableofcontents
\end{frame}

\section{Integration]}

\begin{frame}[fragile]{Integration}
We want to evaluate numerically the integral
\begin{equation}
Q = \int_a^b f(x) dx\,\, ,
\end{equation}
where $f(x)$ may be a well behaved analytical  function or a function given as discrete data $f(x_i)$.
\end{frame}



\section{Integration based on Piecewise Polynomial Interpolation}

\begin{frame}[fragile]{Integration based on Piecewise Polynomial Interpolation}
Suppose that we know the values of $f(x)$ at a finite set of points
(nodes) $\{x_j\}$, $(j=0,\cdots,n)$,  in the interval $[a,b]$. \\
\pause
\bigskip

It is possible to replace $f(x)$ with an interpolated polynomial $p(x)$ whose analytic
integral is known.\\
\pause
\bigskip

This integration, based on interpolation polynomials, is
generally called \textit{Newton-Cotes quadrature formulae}.
\end{frame}

\begin{frame}[fragile]{Integration based on Piecewise Polynomial Interpolation}

Now, consider that the full interval over which
we intend to integrate can be broken down into $N$ sub-intervals
$[a_i,b_i]$ that encompass $N+1$ nodes $x_i$ ($i=0,\cdots,N$) at
which we know the integrand $f(x)$.\\
\pause
\bigskip

The full integral $Q$ can be written as the sum of the sub-integrals $Q_i$:
\begin{equation}
Q = \sum_{i=0}^{N-1} Q_i = \sum_{i=0}^{N-1} \int_{a_i}^{b_i} f(x) dx\,\,.
\end{equation}
\pause

There are many ways to evaluate these sub-integrals.
\end{frame}

\subsection{MidPoint Rule}
\begin{frame}[fragile]{MidPoint Rule}
The simplest approximation is to assume that the function $f(x)$ is
constant on the interval $[a_i,b_i]$.\\
\pause
\bigskip

Hence, using its central value we write
\begin{equation}
Q_i = \int_{a_i}^{b_i} f(x) dx = (b_i-a_i)\, f\left(\frac{a_i+b_i}{2}\right)\,\,.
\end{equation}
\pause

This is also called the ``rectangle rule'' and is an open Newton-Cotes
formula.  \\
\pause
\bigskip

Note that we need to be able to evaluate $f(x)$ at the
midpoint (knowing the value or using an interpolating polynomial).
\end{frame}

\begin{frame}[fragile]{MidPoint Rule}
The error in the midpoint quadrature can be estimated using a Taylor expansion
about the midpoint $m_i = (a_i + b_i) / 2$,
\begin{equation}
f(x) = f(m_i) + f'(m_i)(x-m_i) + 
\frac{f''(m_i)}{2}(x-m_i)^2 + \frac{f'''(m_i)}{6}(x-m_i)^3 + ...
\end{equation}
Integrating this expression from $a_i$ to $b_i$, the odd-order terms drop
out, and what is left is
\begin{equation}
Q_i = \int_{a_i}^{b_i} f(x)dx = f(m_i)(b_i-a_i) +
\frac{f''(m_i)}{24}(b_i-a_i)^3 + ...\,
\end{equation}
\begin{equation}
Q_i = \int_{a_i}^{b_i} f(x)dx = f(m_i)(b_i-a_i) +
\mathcal{O}(h^3) + ...\,
\end{equation}
where $h = (b_i -a_i) \approx \frac{(b-a)}{N}$. 
\end{frame}

\subsection{Trapezoidal Rule}
\begin{frame}[fragile]{Trapezoidal Rule}
This method approximates $f(x)$ with a linear polynomial on $[a_i,b_i]$,
\begin{equation}
Q_i = \int_{a_i}^{b_i} f(x) dx = \frac{1}{2}(b_i - a_i)\left[ f(b_i) + f(a_i) \right] \,\,.
\end{equation}
\pause

Using a Taylor expansion it is shown that the trapezoidal rule has a local error that, like
the midpoint rule, goes with $h^3$. 
\end{frame}

\subsection{Simpson's Rule}
\begin{frame}[fragile]{Simpson's Rule}
Simpson's Rule approximates $f(x)$ on $[a_i,b_i]$ by a polynomial of degree 2. \\
\pause
\bigskip

Writing out the interpolating
polynomial  and integrating we get
\begin{equation}
Q_i = \frac{b_i-a_i}{6} \left[f(a_i) + 4f\left(\frac{a_i+b_i}{2}\right) + f(b_i) \right] + \mathcal{O}([b_i-a_i]^5)\,\,.
\end{equation}
 \pause
 
Note that it is needed the value $f((a_i + b_i)/2)$ and that a special care must be taken at the boundaries.\\

\pause
\bigskip

This method has an error going with $h^5$
\end{frame}


\subsection{Simpson's Rule (non-equidistant intervals)}
\begin{frame}[fragile]{Simpson's Rule (non-equidistant intervals)}
If $f(x)$ is non-equidistantly sampled, i.e.  $h =
[a_i,b_i]$ is not constant, Simpson's rule modifies.
\pause
\bigskip

Write
\begin{equation}
f(x) = a x^2 + b x + c\hspace{2em} \text{for } x \in [x_{i-1},x_{i+1}]\,\,,
\end{equation}
where,
\begin{equation}
\begin{aligned}
a &= \frac{h_{i-1} f(x_{i+1}) - (h_{i-1}+h_i) f(x_{i}) + h_i f(x_{i-1})}
{h_{i-1} h_i(h_{i-1} + h_i)}\,\,,\\
b & = \frac{h_{i-1}^2 f(x_{i+1}) + (h_i^2 - h_{i-1}^2) f(x_i) - h_i^2 f(x_{i-1})}
{h_{i-1} h_i (h_{i-1} + h_i)}\,\,,\\
c & = f(x_i)\,\,,
\end{aligned}
\end{equation}
with $h_i = x_{i+1} - x_i$ and $h_{i-1} = x_i - x_{i-1}$. 
\end{frame}


\begin{frame}[fragile]{Simpson's Rule (non-equidistant intervals)}
Since the integral
\begin{equation*}
\int_{x_{i-1}}^{x_{i+1}} f(x) dx
\end{equation*}
is independent of the origin of the coordinates, we can choose $x_i = 0$, and hence $-h_{i-1} = x_{i-1}$, $h_i = x_{i+1}$ to obtain
\begin{equation}
\int_{x_{i-1}}^{x_{i+1}} f(x) dx = \int_{-h_{i-1}}^{h_i} f(x) dx = \alpha f(x_{i+1})
+ \beta f(x_i) + \gamma f(x_{i-1})\,\,,
\end{equation}
with 
\begin{equation}
\begin{aligned}
\alpha &= \frac{2 h_i^2 + h_i h_{i-1} - h_{i-1}^2}{6 h_i}\,\,, 
&\beta = \frac{ (h_i + h_{i-1})^3 }{6 h_i h_{i-1}}\,\,,\\
\gamma &= \frac{-h_i^2 + h_i h_{i-1} + 2 h_{i-1}^2}{6 h_{i-1}} \; ,
%% \gamma &= \frac{-h_i^2 + h_i h_{i-1} - 2 h_{i-1}^2}{6 h_i}\,\,.
\end{aligned}
\end{equation}
\end{frame}

\section{Gaussian Quadrature}
\begin{frame}[fragile]{Gaussian Quadrature}
Gaussian quadrature is an integration method in which there are chosen two quantities, nodes
and weights, to maximize the degree of the
resulting integration rule.\\
\pause
\bigskip
 
Using $2n$ degrees of freedom, it is possible to integrate
polynomials of degree $2n-1$. The condition is that, for a set of
$n$ nodes $x_i$ and $n$ weights $w_i$ at which a function $f$ is known,

\begin{equation}
\label{eq:gaussian1}
\int_a^b f(x) dx \approx \sum_{i=1}^{n} w_i f(x_i)
\end{equation}
\end{frame}

\begin{frame}[fragile]{2-point Gaussian Quadrature}
In the $2$-point Gaussian quadrature we consider $2$ nodes $x_i$ and $2$ weights $w_i$ on the interval $[a,b]$ to integrate exactly a polynomial of degree $3$. Hence
\begin{equation}
\int_a^b f(x) dx = w_1 f(x_1) + w_2 f(x_2).
\end{equation}
We can find the unknowns 
$w_1$, $w_2$, $x_1$, and $x_2$ by demanding that the
formula give exact results for integrating a general polynomial of
degree $3$,
\small
\begin{equation}
\begin{aligned}
\label{eq:gq1}
\int_a^b f(x) dx & = \int_a^b (c_0 + c_1 x + c_2 x^2 + c_3 x^3) dx\,\,,\\
& = c_0 (b-a) + c_1 \left(\frac{b^2 - a^2}{2} \right)
+ c_2 \left(\frac{b^3 - a^3}{3} \right) + c_3 \left(\frac{b^4 -a^4}{4} \right).
\end{aligned}
\end{equation}
\end{frame}

\begin{frame}[fragile]{2-point Gaussian Quadrature}
However, we have
\footnotesize
\begin{eqnarray}
\int_a^b f(x) dx &=& w_1 f(x_1) + w_2 f(x_2) \notag \\
&=& w_1(c_0 + c_1 x_1 + c_2 x_1^2 + c_3 x_1^3)
+ w_2 (c_0 + c_1 x_2 + c_2 x_2^2 + c_3 x_2^3).\notag
\end{eqnarray}
\normalsize
Therefore, comparing these results gives
\small
\begin{align}
w_1 &= \frac{b-a}{2}\,\,, \hspace*{1.5em} w_2 = \frac{b-a}{2}\,\,,\\
x_1 &= \left( \frac{b-a}{2}\right) \left(-\frac{1}{\sqrt{3}} \right) + 
\left(\frac{b+a}{2} \right)\,\,,\\
x_2 &= \left( \frac{b-a}{2}\right) \left(\frac{1}{\sqrt{3}} \right) + 
\left(\frac{b+a}{2} \right)\,\,.
\end{align}
\end{frame}

\begin{frame}[fragile]{2-point Gaussian Quadrature}
In the special case of the interval $[-1,1]$:
\begin{equation}
w_1 = w_2 = 1\,\,,\hspace{1.5em} x_1 = -\frac{1}{\sqrt{3}}\,\,,\hspace{1.5em}
x_2 = \frac{1}{\sqrt{3}}\,\,,
\end{equation}
and
\begin{equation}
\int_{-1}^1 f(x) dx = f\left(-\frac{1}{\sqrt{3}}\right) + f\left(\frac{1}{\sqrt{3}}\right)\,\,. 
\end{equation}
\end{frame}

\subsubsection{Change of Interval}
\begin{frame}[fragile]{Change of Interval}
Using an affine transformation that maps $[a,b]
\rightarrow [-1,1]$ as 
\begin{equation}
t = \frac{b-a}{2} x + \frac{a+b}{2}\,\,,
\end{equation}
and a change of variables
\begin{equation}
dt = \frac{b-a}{2} dx\,\,,
\end{equation}
it is possible to change the interval of integration, 
\begin{equation}
 \int_{a}^{b} f(x) dx = \frac{b-a}{2} \int_{-1}^1 f\left(\frac{b-a}{2}x + \frac{a+b}{2}\right) dx\,\,.
\end{equation}
\end{frame}


\end{document}